%========================================================================================
%   PREAMBLE: PACKAGES AND DOCUMENT SETUP
%========================================================================================

\documentclass[11pt, a4paper]{article}

%----------------------------------------------------------------------------------------
%   CORE PACKAGES
%----------------------------------------------------------------------------------------

\usepackage{amsmath}        % For advanced math environments (align, etc.)
\usepackage{amssymb}        % For math symbols (e.g., \le)
\usepackage{xcolor}         % For defining and using colors
\usepackage{hyperref}       % For creating clickable links (e.g., in the ToC)
\usepackage{pifont}         % For Zapf Dingbats symbols, like arrows

%----------------------------------------------------------------------------------------
%   PAGE LAYOUT & APPEARANCE
%----------------------------------------------------------------------------------------

\usepackage[a4paper, margin=0.8in, headheight=15pt]{geometry} % Sets page margins
\usepackage[most]{tcolorbox}                                 % The core package for all colored boxes
\usepackage{fancyhdr}                                        % For creating custom headers and footers
\usepackage{lastpage}                                        % Required for "Page X of Y" footer format
\usepackage{tabularx}                                        % For tables with auto-wrapping text columns

%----------------------------------------------------------------------------------------
%   HYPERLINK SETUP
%----------------------------------------------------------------------------------------

\hypersetup{
    colorlinks=true,       % Enables colored links instead of boxes
    linkcolor=teal,        % Color for internal links (e.g., Table of Contents)
    urlcolor=teal,         % Color for external URLs
    pdftitle={Ultimate Steel Design Workflow}, % PDF metadata
    pdfauthor={Your Name}, % PDF metadata
}

%----------------------------------------------------------------------------------------
%   CUSTOM TCOLORBOX STYLES
%----------------------------------------------------------------------------------------

% Box for high-level strategic advice
\newtcolorbox{strategybox}{
    colback=gray!5!white,
    colframe=gray!60!black,
    fonttitle=\bfseries,
    title=Strategy
}

% Box for critical decision points in the workflow
\newtcolorbox{decisionbox}{
    colback=green!5!white,
    colframe=green!50!black,
    fonttitle=\bfseries,
    title=DECISION POINT
}

% Box for important notes, warnings, and common pitfalls
\newtcolorbox{notebox}{
    colback=yellow!10!white,
    colframe=yellow!80!black,
    fonttitle=\bfseries,
    title=Important Note
}

% Box for final answers, themed with a custom color
\newtcolorbox{finalbox}[2][]{
    colback=#2!5!white,
    colframe=#2!75!black,
    fonttitle=\bfseries,
    title=Final Answer Check,
    #1
}

%----------------------------------------------------------------------------------------
%   HANDHOLDING MACROS (exam guidance boxes)
%----------------------------------------------------------------------------------------

% Shorthand for fill-in blanks
\newcommand{\blank}[1]{\underline{\hspace{#1}}}

% Keyword highlighter (teal bold)
\newcommand{\kword}[1]{\textbf{\textcolor{teal}{#1}}}

% "DO THIS NOW" action box
\newtcolorbox{dobox}{
    colback=blue!4!white,
    colframe=blue!60!black,
    fonttitle=\bfseries,
    title=DO THIS NOW
}

% Tick-checklist box
\newtcolorbox{checklistbox}{
    colback=gray!6!white,
    colframe=gray!60!black,
    fonttitle=\bfseries,
    title=Tick-Checklist
}

% Sanity check box (catch stupid mistakes)
\newtcolorbox{sanitybox}{
    colback=orange!8!white,
    colframe=orange!80!black,
    fonttitle=\bfseries,
    title=Sanity Check (catch mistakes)
}

% Answer template box
\newtcolorbox{answerbox}{
    colback=green!6!white,
    colframe=green!60!black,
    fonttitle=\bfseries,
    title=Write This In Your Answer
}

%----------------------------------------------------------------------------------------
%   CUSTOM HEADER AND FOOTER SETUP
%----------------------------------------------------------------------------------------

\pagestyle{fancy}                       % Activates the fancy page style
\fancyhf{}                              % Clears all existing header and footer fields
\fancyhead[L]{Ultimate Steel Design: A Decision-Based Workflow (AS 4100)} % Left-aligned header
\fancyfoot[C]{\thepage\ of \pageref{LastPage}} % Centered footer: "Page X of Y"
\renewcommand{\headrulewidth}{0.4pt}    % Adds a rule line below the header
\renewcommand{\footrulewidth}{0.4pt}    % Adds a rule line above the footer

%----------------------------------------------------------------------------------------
%   DOCUMENT INFORMATION (FOR TITLE PAGE)
%----------------------------------------------------------------------------------------

\title{\bfseries \Huge The Ultimate Steel Design Workflow \\ \large A Decision-Based Guide for AS 4100}
\author{Your Name | EGB376 - Steel Structures} % Replace with your details
\date{} % Suppresses the date from being displayed

%========================================================================================
%   END OF PREAMBLE
%========================================================================================


%========================================================================================
%   BEGIN DOCUMENT
%========================================================================================

\begin{document}

%----------------------------------------------------------------------------------------
%   TITLE PAGE
%----------------------------------------------------------------------------------------
\maketitle
\thispagestyle{fancy} % Apply the custom header/footer to the title page
\hrule
\vspace{1em}
This document is an interactive, step-by-step procedure to solve steel design problems according to AS 4100. It is designed as a plug-and-play checklist for open-book assessments. Start every problem with the Master Decision Tree in Section 2.
\vspace{1em}
\tableofcontents

\newpage

%----------------------------------------------------------------------------------------
%   SECTION 0.5: ADVANCED LOAD ANALYSIS (UNNUMBERED - DOES NOT SHIFT SECTION NUMBERS)
%----------------------------------------------------------------------------------------
\section*{Advanced Load Analysis: Patterns \& Combinations}
\addcontentsline{toc}{section}{Advanced Load Analysis: Patterns \& Combinations}
\begin{strategybox}
Before calculating capacity, find the worst-case Design Action ($S^*$). For continuous beams, use \textbf{pattern loading}.
\end{strategybox}
\subsection*{Pattern Loading Rules}
\begin{itemize}
    \item \textbf{Dead Load ($G$):} Apply to the \textbf{entire} structure.
    \item \textbf{Live ($Q$) \& Snow ($S$):} Apply only where they increase the effect.
    \begin{itemize}
        \item \textbf{Max hogging at support:} Load adjacent spans. Unload far spans.
        \item \textbf{Max sagging in span:} Load that span. Unload adjacent spans.
    \end{itemize}
\end{itemize}
\subsection*{Critical Combinations Checklist}
\begin{enumerate}
    \item \textbf{Gravity:} $1.2G + 1.5Q$
    \item \textbf{Snow:} $1.2G + S_u + \psi_c Q$
    \item \textbf{Gravity + Wind (downwards):} $1.2G + W_u + \psi_c Q$
    \item \textbf{Uplift / Overturning (Stability):} $0.9G + W_{up}$ (Often assume $Q=0$ for uplift.)
\end{enumerate}
\begin{notebox}
\textbf{Do not guess combinations:} Use AS/NZS 1170.0 combination sets relevant to the action \textbf{and direction}. Always check uplift/overturning/sliding separately (often with reduced $G$ and destabilising actions).
\end{notebox}

\newpage

%----------------------------------------------------------------------------------------
%   SECTION 1: FUNDAMENTALS & FIRST STEPS
%----------------------------------------------------------------------------------------

\section*{How to Use This Document}
\begin{tcolorbox}[colback=gray!5, colframe=gray!50!black, fonttitle=\bfseries, title=Your Problem-Solving Process]
For every exam question, follow this exact procedure:
\begin{enumerate}
    \item \textbf{Start at the Master Decision Tree (Section 2).} Answer the questions to identify the correct workflow for your problem.
    \item \textbf{Go to the indicated workflow section.}
    \item \textbf{Follow the steps and decision points within that workflow.} Fill in the blanks as you go.
    \item This process ensures you never miss a critical check.
\end{enumerate}
\end{tcolorbox}

\vspace{0.5em}
\section*{Question Decoder (60-second triage)}
\begin{strategybox}
Use this decoder to translate exam wording into \textbf{(i)} what you must produce, \textbf{(ii)} which limit state you are in, and \textbf{(iii)} the workflow section to use. Read the question once, then fill the outputs below.
\end{strategybox}

\begin{tcolorbox}[colback=gray!5, colframe=gray!50!black, fonttitle=\bfseries, title=Decoder Output (fill these first)]
\begin{tabularx}{\textwidth}{@{}lX@{}}
\textbf{You must produce:} & \underline{\hspace{10cm}} \\
\textbf{Limit state:} & \underline{\hspace{10cm}} \quad (Strength / Serviceability) \\
\textbf{Primary action(s):} & \underline{\hspace{10cm}} \quad ($N^*$ / $M^*$ / $V^*$ / combined) \\
\textbf{Member vs connection:} & \underline{\hspace{10cm}} \quad (Member / Connection / Both) \\
\textbf{Special flags:} & \underline{\hspace{10cm}} \quad (slender / unbraced / holes / eccentricity / built-up) \\
\end{tabularx}
\end{tcolorbox}

\begin{notebox}
\textbf{STOP 1 (Loads):} Have you found the governing ULS/SLS combination \textbf{and} worst-case pattern loading case (if continuous)?
\end{notebox}
\begin{notebox}
\textbf{STOP 2 (Restraint):} Is the member restrained? Which flange is in compression at the critical section?
\end{notebox}
\begin{notebox}
\textbf{STOP 3 (Weakest link):} Have you checked member + connection + local effects (web bearing, holes, eccentricity)?
\end{notebox}

\begin{decisionbox}
\textbf{Step A: Decode the verb (what is being asked?).}
\begin{itemize}
    \item \textbf{IF the question says} ``check'', ``adequate?'', ``verify'', ``PASS/FAIL'' \textbf{THEN} you are doing an \textbf{adequacy check:} compute actions ($S^*$) and compare to capacities ($\phi R$).
    \item \textbf{IF the question says} ``calculate the capacity'', ``determine the maximum'', ``find $\phi R$'' \textbf{THEN} you are doing a \textbf{capacity calculation:} compute $\phi R$ directly (often several checks) and take the minimum governing value.
    \item \textbf{IF the question says} ``select a section'', ``design a member'', ``size the weld'', ``number of bolts'' \textbf{THEN} you are doing a \textbf{sizing problem:} rearrange so capacity $\ge$ demand, then pick the smallest practical option.
\end{itemize}
\end{decisionbox}

\begin{decisionbox}
\textbf{Step B: Decide the limit state (which loads?).}
\begin{itemize}
    \item \textbf{IF the question mentions} strength, yielding, buckling, fracture, ultimate, collapse \textbf{THEN} use \textbf{Strength} combinations (factored).
    \item \textbf{IF the question mentions} deflection, vibration, serviceability, comfort \textbf{THEN} use \textbf{Serviceability} combinations (unfactored/service).
\end{itemize}
\end{decisionbox}

\begin{tcolorbox}[colback=gray!5, colframe=gray!50!black, fonttitle=\bfseries, title=Keyword $\rightarrow$ Workflow Map (fast navigation)]
\renewcommand{\arraystretch}{1.25}
\begin{tabularx}{\textwidth}{@{}X X l@{}}
\textbf{If the prompt looks like...} & \textbf{Your first move is...} & \textbf{Go to} \\
\hline
Axial tension member / tie / plate in tension; net area, block shear, fracture & Compute $N^*$ then check member + connection (weakest link governs) & Sec. 3 \\
Column / strut / compression member; buckling, slenderness, effective area & Compute $N^*$; check section capacity + member buckling & Sec. 4 \\
Beam; bending + shear + web bearing; LTB / unbraced length & Compute $M^*, V^*$; check bending then shear then web local checks & Sec. 5 \\
Beam-column; combined $N^*$ and $M^*$; interaction check & Gather pure capacities first, then apply interaction checks & Sec. 6 \\
Eccentric welded bracket/plate; weld group under shear + moment & Resolve direct + torsional shear in weld group; check resultant vs capacity & Sec. 7 \\
Bolts in tension and shear; combined action on most critical bolt & Compute per-bolt actions; check shear, tension, then interaction & Sec. 8 \\
\end{tabularx}
\end{tcolorbox}

\begin{notebox}
\textbf{Common trap flags (read the diagram/details):}
\begin{itemize}
    \item \textbf{Built-up/custom section or slender plates:} you may need \textbf{effective} properties ($A_e$, $I_{eff}$, $Z_e$), not tabulated values.
    \item \textbf{Unbraced beam segment:} LTB may govern (capacity reduced).
    \item \textbf{Bolt holes / stagger:} net area path and block shear may govern.
    \item \textbf{Eccentricity:} identify torsion/moment on weld/bolt group (not just direct shear).
\end{itemize}
\end{notebox}


\section{Fundamentals of Limit States Design}

\begin{notebox}
\textbf{Quick Reference Constants:}
\begin{itemize}
    \item Capacity Factors: \textbf{See Table 3.4.} (Depends on the limit state and connection category.) Commonly, $\phi = 0.9$ for many member checks and $\phi = 0.8$ for bolts and SP welds.
    \item Material Properties: Steel Density $\rho = 7850$ kg/m$^3$, $E = 200,000$ MPa, $G = 80,000$ MPa.
\end{itemize}
\end{notebox}
\begin{notebox}
\textbf{Units Guardrail (don’t lose marks):}
\begin{itemize}
    \item $1$ MPa $= 1$ N/mm$^2$
    \item $A$ (mm$^2$) $\cdot f$ (MPa) $=$ N
    \item N $\rightarrow$ kN: divide by $1000$
    \item Nmm $\rightarrow$ kNm: divide by $10^6$
\end{itemize}
\end{notebox}

\begin{strategybox}
Structural design ensures safety against uncertain loads and material strengths. Limit States Design manages this uncertainty using two key safety factors:
\begin{enumerate}
    \item \textbf{Load Factors ($\gamma$):} We \textbf{increase} the specified loads (G, Q, W) to get a conservative "Design Action Effect" ($S^*$). This accounts for potential \textbf{overloading}.
    \item \textbf{Capacity Reduction Factors ($\phi$):} We \textbf{decrease} the theoretical strength of a member to get a "Design Capacity" ($\phi R$). This accounts for potential \textbf{under-strength} and variability.
\end{enumerate}
The structure is safe if the factored action is less than or equal to the factored capacity.
\end{strategybox}

\subsection*{The Core Design Equation}
\begin{tcolorbox}[colback=gray!5, colframe=gray!50!black, sharp corners, center]
\Huge $S^* \le \phi R$
\end{tcolorbox}

\subsection{Step 1 of Every Problem: Calculate the Design Action Effect ($S^*$)}
Before using any workflow, you must determine the target load ($N^*, M^*, V^*$) your member needs to resist. This is calculated from the nominal loads given in the question using the appropriate load combinations.

\subsubsection{Strength Limit State Load Combinations (AS/NZS 1170.0)}
\begin{notebox}
For any strength problem (checking capacity for yielding, fracture, buckling), you must test all relevant load combinations and use the one that produces the \textbf{WORST} (i.e., largest) design action effect.
\end{notebox}
\begin{itemize}
    \item \textbf{For Downward Gravity Loads (most common case):}
        \[ S^* = 1.2G + 1.5Q \]
        Where: $G$ = Permanent Action (Dead Load), $Q$ = Imposed Action (Live Load).

    \item \textbf{For Wind Uplift or Reversal (when wind opposes gravity):}
        \[ S^* = 0.9G + W_u \]
        Where: $W_u$ = Ultimate Wind Action.

    \item \textbf{Permanent action dominant (G-only ULS case, if applicable):}
        \[ S^* = 1.35G \]
        \textit{Used only when variable actions are absent or demonstrably negligible for the limit state under consideration.}
        
        \textit{Note: Many "stability" questions actually mean uplift/overturning/sliding, where reduced $G$ and destabilising actions govern.}
        
    \item \textbf{For Gravity with Wind (downwards):}
        \[ S^* = 1.2G + W_u + \psi_c Q \]
        Where $\psi_c$ is the combination factor for the imposed action.
\end{itemize}
\begin{notebox}
\textbf{Do not guess combos:} Use AS/NZS 1170.0 combination sets relevant to the action \textbf{and direction}; check uplift/overturning separately.
\end{notebox}

\subsubsection{Serviceability Limit State (Checks for Deflection, Vibration)}
\begin{notebox}
Serviceability is about performance and comfort, not collapse. We use \textbf{unfactored} (or service) loads because we are checking behaviour under normal, everyday conditions.
\end{notebox}
\begin{notebox}
\textbf{$\psi$ factors: check Table 4.1 of AS/NZS 1170.0 first.}
\begin{itemize}
    \item The short-term ($\psi_s$), long-term ($\psi_l$), and combination ($\psi_c$) factors depend on the action category / occupancy.
    \item \textit{Last-resort memory aid only (still confirm):} often $\psi_s \approx 0.7$ and $\psi_c \approx 0.4$, but these can be different (including $0.0$).
\end{itemize}
\end{notebox}
\begin{itemize}
    \item \textbf{Short-Term Effects (e.g., immediate deflection):}
        \[ S_{ser} = G + \psi_s Q \]
    \item \textbf{Long-Term Effects (e.g., creep, settlement):}
        \[ S_{ser} = G + \psi_l Q \]
    \item \textbf{Wind Serviceability (when checking wind deflection/drift):}
        \[ S_{ser} = G + W_s \quad \text{or} \quad S_{ser} = W_s \]
        Where: $W_s$ = Serviceability Wind Action (as specified in the question/standard).
    \item The short-term ($\psi_s$) and long-term ($\psi_l$) factors are found in \textbf{Table 4.1 of AS/NZS 1170.0}. They depend on the use of the structure (e.g., residential, storage).
\end{itemize}

\newpage

%========================================================================================
%   END OF SECTION 1
%========================================================================================

%========================================================================================
%   SECTION 2: MASTER DECISION TREE
%========================================================================================

\section{Master Decision Tree: START EVERY PROBLEM HERE}

\begin{strategybox}
Every steel design problem can be categorized by the primary actions on the member and the limit state being checked. This decision tree is a triage system. Read the exam question carefully, then answer the questions below to find the exact workflow you need to solve it.
\end{strategybox}

\begin{decisionbox}
\textbf{1. What is the PRIMARY action the member is designed for?}

\begin{itemize}
    \item \textbf{IF} the member is primarily resisting \textbf{Axial Tension} (being pulled apart by a force $N^*$), and the question asks for its design capacity...
        \begin{itemize}
            \item[\ding{226}] \textbf{Go to Section 3: Tension System Workflow}
        \end{itemize}

    \item \textbf{IF} the member is primarily resisting \textbf{Axial Compression} (a column being squashed by a force $N^*$), and the question asks for its design capacity...
        \begin{itemize}
            \item[\ding{226}] \textbf{Go to Section 4: Compression Member Workflow}
        \end{itemize}

    \item \textbf{IF} the member is primarily resisting \textbf{Bending} (a beam supporting transverse loads, creating a moment $M^*$), and the question asks for its capacity or adequacy...
        \begin{itemize}
            \item[\ding{226}] \textbf{Go to Section 5: Flexural System (Beam) Workflow}
        \end{itemize}

    \item \textbf{IF} the member is resisting both \textbf{Axial Compression AND Bending} simultaneously (a beam-column with forces $N^*$ and $M^*$), and the question asks for its adequacy...
        \begin{itemize}
            \item[\ding{226}] \textbf{Go to Section 6: Combined Actions (Beam-Column) Workflow}
        \end{itemize}
\end{itemize}

\vspace{1em}\hrule\vspace{1em}

\textbf{2. Is the question ONLY about a complex connection detail?}

\begin{itemize}
    \item \textbf{IF} the question shows a bracket or plate eccentrically \textbf{welded} to a column and asks for the adequacy of the weld...
        \begin{itemize}
            \item[\ding{226}] \textbf{Go to Section 7: Advanced Connection --- Eccentric Welds}
        \end{itemize}

    \item \textbf{IF} the question shows a \textbf{bolted} connection resisting pure Tension ($N_{tf}^*$) or a combination of Tension and Shear ($N_{tf}^*$ and $V_f^*$), and asks for the adequacy of the bolts...
        \begin{itemize}
            \item[\ding{226}] \textbf{Go to Section 8: Advanced Connection --- Bolts in Tension \& Combined Action}
        \end{itemize}
\end{itemize}
\end{decisionbox}

\newpage

%========================================================================================
%   END OF SECTION 2
%========================================================================================

%========================================================================================
%   SECTION 3: TENSION SYSTEM WORKFLOW
%========================================================================================

\section{Tension System Workflow ($N^* \le \phi N_{system}$)}

\begin{strategybox}
A tension system can fail in two places: in the main body of the member, or at the connection. The true capacity of the system is the \textbf{weakest link} in this chain. This workflow ensures you check all possible failure modes for both the member and its connection to find the governing capacity.
\end{strategybox}

\begin{checklistbox}
\textbf{If the question says... then you MUST include...}
\begin{itemize}
    \item \kword{bolt holes} / \kword{net area} $\Rightarrow$ CHECK 2 (Net Section Fracture) with correct $d_h$
    \item \kword{staggered} / \kword{zig-zag} $\Rightarrow$ Check ALL failure paths using $s_p^2/(4s_g)$ correction
    \item \kword{block shear} / \kword{tear-out} $\Rightarrow$ CHECK 5 (Block Shear) is mandatory
    \item \kword{angle} / \kword{channel} / \kword{one leg connected} $\Rightarrow$ use $k_t < 1.0$ (Table 7.3.2)
    \item \kword{welded} $\Rightarrow$ skip bolt checks, do weld capacity + parent metal
\end{itemize}
\end{checklistbox}

\begin{dobox}
\textbf{Before you start calculations:}
\begin{enumerate}
    \item Write the governing ULS load combination: \blank{5cm}
    \item Calculate $N^* = $ \blank{3cm} kN
    \item Is it bolted or welded? \blank{3cm}
    \item Are there holes in the critical cross-section? \underline{YES / NO}
    \item Is the member connected on all legs/flanges, or only one? \blank{4cm}
\end{enumerate}
\end{dobox}

%----------------------------------------------------------------------------------------
%   PART A: MEMBER CAPACITY
%----------------------------------------------------------------------------------------

\subsection{Part A: Member Capacity Checks}
\begin{tcolorbox}[colback=gray!5,colframe=gray!50,title=Phase 1: Setup \& Data Extraction]
    \begin{itemize}
        \item \textbf{Member/Grade:} \underline{\hspace{8cm}}
        \item \textbf{Properties from Tables (AS 4100 \& Steel Handbook):}
        \begin{itemize}
            \item Capacity Factor (Yielding \& Fracture) $\phi = 0.9$ (Table 3.4)
            \item Yield Stress $f_y = \underline{\hspace{2cm}}$ MPa (Table 2.1)
            \item Ultimate Tensile Strength $f_u = \underline{\hspace{2cm}}$ MPa (Table 2.1)
            \item Gross Area $A_g$ (of one section) = \underline{\hspace{2cm}} mm$^2$
            \item Thickness of connected part $t = \underline{\hspace{2cm}}$ mm
        \end{itemize}
    \end{itemize}
\end{tcolorbox}

\begin{notebox}
\textbf{Manual Area Check (if no tables):} \\
For Angles: $A_g \approx (b_1 + b_2 - t) \times t$. \\
For Tees: Sum the rectangles.
\end{notebox}

\subsubsection*{CHECK 1: Gross Section Yielding (Clause 7.2)}
\begin{itemize}
    \item \textbf{Purpose:} Checks for ductile stretching of the member away from connections.
    \item \textbf{Formula:} $\phi N_y = \phi A_g f_y$
\end{itemize}
\begin{dobox}
\textbf{Write the working exactly like this:}
\[
\phi N_y = 0.9 \times A_g \times f_y = 0.9 \times (\blank{2.5cm}\,\text{mm}^2) \times (\blank{2cm}\,\text{MPa}) = \blank{2.5cm}\,\text{kN}
\]
\textbf{Result T1 = \blank{3cm} kN}
\end{dobox}
\begin{sanitybox}
$A(\text{mm}^2) \times f(\text{MPa}) = \text{N}$. Divide by 1000 to get kN. \\
\textit{If your answer is millions of kN, you forgot to convert.}
\end{sanitybox}

\hrule\vspace{1em}

\subsubsection*{CHECK 2: Net Section Fracture (Clause 7.2)}
\begin{itemize}
    \item \textbf{Purpose:} Checks for brittle failure at the weakened cross-section through bolt holes.
    \item \textbf{Formula:} $\phi N_u = \phi \times k_t \times A_n \times f_u$
    \item \textbf{Step a) Calculate Net Area ($A_n$):}
        \begin{notebox}
        \textbf{Common Pitfall:} Always use the diameter of the \textbf{hole}, not the bolt. \\
        Hole Diameter ($d_h$) = Bolt Diameter ($d_f$) + 2 mm (for bolts up to M24).
        \textit{If provided, use the standard hole size table (preferred over memory rules).}
        \end{notebox}
        \begin{itemize}
            \item Bolt $d_f = \underline{\hspace{1.5cm}}$ mm $\implies$ Hole $d_h = \underline{\hspace{1.5cm}}$ mm
            \item $A_n = A_g - (\text{No. holes in cross-section}) \times d_h \times t = \underline{\hspace{2cm}}$ mm$^2$
            \item \textit{For staggered connections (check all paths):} $A_n = A_g - n(d_h t) + \sum \frac{s_p^2 t}{4s_g}$
        \end{itemize}
    \item \textbf{Step b) Determine Correction Factor ($k_t$) (Table 7.3.2):}
        \begin{itemize}
            \item Selected $k_t = \underline{\hspace{2cm}}$ (e.g., 1.0 for plates, 0.85 for angles or I-sections connected by flanges only)
            \item \textbf{Welded Connections:}
                \begin{itemize}
                    \item Welded on one side only (e.g., flange only): $k_t = 0.85$.
                    \item Welded all around (balanced): $k_t = 1.0$.
                \end{itemize}
        \end{itemize}
    \item \textbf{Step c) Calculate Fracture Capacity:}
        \begin{align*}
            \phi N_u &= (0.9) \times (\underline{\hspace{1cm}}) \times (\underline{\hspace{2cm}}\,\text{mm}^2) \times (\underline{\hspace{2cm}}\,\text{MPa}) \\
            &= \underline{\hspace{4cm}}\,\text{kN} \quad \rightarrow \quad \textbf{(Result T2)}
        \end{align*}
\end{itemize}
\begin{tcolorbox}[colback=blue!5, colframe=blue!75, title=\textbf{Member Capacity Result}]
    The governing capacity of the member itself is $\phi N_{member} = \min(\text{Result T1, Result T2}) = \underline{\hspace{3cm}}$ kN.
\end{tcolorbox}

%----------------------------------------------------------------------------------------
%   PART B: CONNECTION CAPACITY
%----------------------------------------------------------------------------------------

\subsection{Part B: Connection Capacity Checks}

\begin{decisionbox}
\textbf{What type of connection is used in the problem?}
\begin{itemize}
    \item \textbf{IF} the members are joined using \textbf{bolts}...
        \begin{itemize}
            \item[\ding{226}] \textbf{Proceed to Part B1: Bolted Connection Checks}
        \end{itemize}
    \item \textbf{IF} the members are joined using \textbf{welds}...
        \begin{itemize}
            \item[\ding{226}] \textbf{Proceed to Part B2: Welded Connection Checks}
        \end{itemize}
\end{itemize}
\end{decisionbox}

\begin{notebox}
\textbf{Splice/Splice-Plate Reminder:} If the member is spliced, the \textbf{system} capacity may be governed by the splice detail. Check the connection actions and the splice plate/member net section and block shear as applicable. The weakest link still governs.
\end{notebox}

\subsubsection*{Part B1: Bolted Connection Checks}
\begin{itemize}
    \item \textbf{CHECK 3: Bolt Shear Capacity ($\phi V_f$)} (Clause 9.3.2.1)
        \begin{itemize}
            \item \textbf{Formula:} $\phi V_f = \phi (0.62 f_{uf} A_{eff} k_r)$
            \item Capacity of one bolt in shear = $\underline{\hspace{2cm}}$ kN
            \item Total bolt shear capacity = (Capacity of one bolt) $\times$ (No. shear planes) $\times$ (No. bolts)
            \item Total $\phi V_{f,total} = \underline{\hspace{3cm}}$ kN $\quad \rightarrow \quad \textbf{(Result BC1)}$
        \end{itemize}
    \item \textbf{CHECK 4: Ply Bearing Capacity ($\phi V_b$)} (Clause 9.3.2.4)
        \begin{itemize}
            \item \textbf{Formula:} $V_b = \min(3.2 d_f t_p f_{up}, \ a_e t_p f_{up})$
            \item $V_b$ per bolt in weakest ply = $\underline{\hspace{2cm}}$ kN
            \item Total ply bearing capacity = ($V_b$ per bolt) $\times$ (No. bolts)
            \item Total $\phi V_{b,total} = 0.9 \times V_{b,total} = \underline{\hspace{3cm}}$ kN $\quad \rightarrow \quad \textbf{(Result BC2)}$
        \end{itemize}
    \item \textbf{CHECK 5: Block Shear Capacity ($\phi V_{bs}$)} (Clause 9.3.3)
        \begin{itemize}
            \item \textbf{Formula:} $\phi V_{bs} = \phi \times \min[ (0.6 f_u A_{nv} + f_y A_{gt}), (0.6 f_y A_{gv} + f_u A_{nt}) ]$
            \item $\phi = 0.9$ (for connections)
            \item $\phi V_{bs} = \underline{\hspace{3cm}}$ kN $\quad \rightarrow \quad \textbf{(Result BC3)}$
        \end{itemize}
\end{itemize}
\begin{tcolorbox}[colback=violet!5, colframe=violet!75, title=\textbf{Bolted Connection Capacity Result}]
    The governing capacity of the bolted connection is $\phi N_{conn} = \min(\text{Result BC1, BC2, BC3}) = \underline{\hspace{3cm}}$ kN.
\end{tcolorbox}


\subsubsection*{Part B2: Welded Connection Checks}
\begin{itemize}
    \item \textbf{CHECK 3: Weld Group Capacity ($\phi N_w$)} (Clause 9.7)
        \begin{itemize}
            \item \textbf{Formula:} $\phi N_w = \phi \times (0.6 f_{uw} t_t) \times L_{total}$
            \item Weld throat thickness $t_t = 0.707 \times t_{w,leg} = \underline{\hspace{2cm}}$ mm
            \item Total length of weld $L_{total} = \underline{\hspace{2cm}}$ mm
            \item Capacity factor for weld $\phi = \underline{\hspace{1cm}}$ (e.g., 0.8 for SP)
            \item $\phi N_w = \underline{\hspace{3cm}}$ kN $\quad \rightarrow \quad \textbf{(Result W1)}$
        \end{itemize}
    \item \textbf{CHECK 4: Parent Metal Capacity}
        \begin{itemize}
            \item The connection is also limited by the strength of the material next to the weld. This is typically the Gross Section Yielding of the connected plate.
            \item Parent Metal Capacity = (Result T1 from Part A) = $\underline{\hspace{3cm}}$ kN $\quad \rightarrow \quad \textbf{(Result W2)}$
        \end{itemize}
\end{itemize}
\begin{tcolorbox}[colback=teal!5, colframe=teal!75, title=\textbf{Welded Connection Capacity Result}]
    The governing capacity of the welded connection is $\phi N_{conn} = \min(\text{Result W1, W2}) = \underline{\hspace{3cm}}$ kN.
\end{tcolorbox}

%----------------------------------------------------------------------------------------
%   PART C: FINAL CONCLUSION
%----------------------------------------------------------------------------------------

\subsection{Part C: Final Conclusion for the Tension System}
\begin{finalbox}{blue}
To find the design capacity of the entire system, compare the governing member capacity (Part A) with the governing connection capacity (Part B). The weakest link governs.
\begin{itemize}
    \item Governing Member Capacity ($\phi N_{member}$) = \underline{\hspace{3cm}} kN
    \item Governing Connection Capacity ($\phi N_{conn}$) = \underline{\hspace{3cm}} kN
\end{itemize}
\textbf{The Design Capacity of the Tension System is $\phi N_{system} = \min(\phi N_{member}, \phi N_{conn}) = \underline{\hspace{4cm}}$ kN.}
\end{finalbox}

\begin{decisionbox}
\textbf{If Net Section or Block Shear FAILS, what changes?}
\begin{itemize}
    \item Increase plate thickness $t$ \quad OR
    \item Reduce number of holes on the critical path \quad OR
    \item Add/extend splice plates \quad OR
    \item Switch to welded connection (removes hole loss)
\end{itemize}
\textbf{Then re-check:} $A_n$, block shear, and connection capacity.
\end{decisionbox}

\begin{answerbox}
\textbf{Copy this structure into your exam answer:}
\begin{enumerate}
    \item Governing combination: \blank{4cm} $\Rightarrow$ $N^* = $ \blank{2cm} kN
    \item Member capacity: $\phi N_{member} = \min(\phi N_y, \phi N_u) = $ \blank{2cm} kN \quad (governs by: \blank{3cm})
    \item Connection capacity: $\phi N_{conn} = $ \blank{2cm} kN \quad (governs by: \blank{3cm})
    \item System capacity: $\phi N_{system} = \min(\blank{2cm}, \blank{2cm}) = $ \blank{2cm} kN
    \item Check: $N^* = $ \blank{2cm} kN $\le \phi N_{system} = $ \blank{2cm} kN $\Rightarrow$ \textbf{PASS / FAIL}
\end{enumerate}
\end{answerbox}

\newpage

%========================================================================================
%   END OF SECTION 3
%========================================================================================

%========================================================================================
%   SECTION 4: COMPRESSION MEMBER WORKFLOW
%========================================================================================

\section{Compression Member Workflow ($N^* \le \phi N_c$)}

\begin{strategybox}
A compression member (column) can fail in two distinct ways. A short, stocky column will crush under load (\textbf{Section Capacity}). A long, slender column will bend and buckle sideways before it reaches its crush strength (\textbf{Member Capacity}). The design capacity is always the \textbf{LOWER} of these two values.
\end{strategybox}

\begin{checklistbox}
\textbf{If the question says... then you MUST include...}
\begin{itemize}
    \item \kword{slender} / \kword{thin plates} / \kword{local buckling} $\Rightarrow$ Calculate $A_e$ using effective widths (Step 0)
    \item \kword{sway} / \kword{portal frame} / \kword{unbraced frame} $\Rightarrow$ Use $k_e > 1.0$ from sway charts
    \item \kword{intermediate bracing} / \kword{restraint at midheight} $\Rightarrow$ Check each segment separately
    \item \kword{welded} / \kword{built-up} $\Rightarrow$ Use HW or LW residual stress category (not HR)
    \item \kword{both axes} / \kword{biaxial} $\Rightarrow$ Calculate $\lambda_n$ for BOTH axes, use the larger
\end{itemize}
\end{checklistbox}

\begin{dobox}
\textbf{Before you start calculations:}
\begin{enumerate}
    \item Write the governing ULS load combination: \blank{5cm}
    \item Calculate $N^* = $ \blank{3cm} kN
    \item Is the frame braced or sway? \blank{3cm}
    \item What is $k_e$ for x-axis? \blank{2cm} \quad for y-axis? \blank{2cm}
    \item Is the section compact or slender (check $b/t$ vs $\lambda_{ey}$)? \blank{3cm}
\end{enumerate}
\end{dobox}

\begin{notebox}
\textbf{Compression Area Reminder:} For compression members, always use
\[ N_s = k_f A_e f_y \]
where $A_e$ is the \textbf{effective area}. $A_e = A_g$ \textbf{only if} the section is non-slender (no local buckling) and has no holes. For slender plates, you must calculate $A_e$ using effective widths.
\end{notebox}

%----------------------------------------------------------------------------------------
%   PART A: SECTION CAPACITY (CRUSHING)
%----------------------------------------------------------------------------------------

\subsection{Part A: Section Capacity Check ($\phi N_s$)}
\begin{tcolorbox}[colback=gray!5,colframe=gray!50,title=Phase 1: Setup \& Data Extraction]
    \begin{itemize}
        \item \textbf{Member/Grade:} \underline{\hspace{8cm}}
        \item \textbf{Properties from Tables (AS 4100 \& Steel Handbook):}
        \begin{itemize}
            \item Capacity Factor (Compression) $\phi = 0.9$ (Table 3.4)
            \item Yield Stress $f_y = \underline{\hspace{2cm}}$ MPa (If multiple grades, use the \textbf{lowest} $f_y$ for the global capacity calc $N_s$.)
            \item \textit{Note: For Local Buckling ($\lambda_e$), check each element using its OWN $f_y$.}
            \item Net Area $A_n = \underline{\hspace{2cm}}$ mm$^2$ \quad \textit{(If no holes: $A_n = A_g$. Do not reduce area unless holes exist.)}
            \item \textit{Slender plates: use $A_e$ (effective area) in CHECK 1, not $A_n$ or $A_g$.}
            \item Form Factor $k_f = \underline{\hspace{2cm}}$ (Clause 6.2.2 - this is 1.0 for compact sections)
            \item Radii of Gyration: $r_x = \underline{\hspace{2cm}}$ mm, $r_y = \underline{\hspace{2cm}}$ mm
            \item Member Length $L = \underline{\hspace{2cm}}$ mm
        \end{itemize}
    \end{itemize}
\end{tcolorbox}

\subsubsection*{CHECK 1: Section Capacity (Yielding/Crushing) (Clause 6.2.1)}
\begin{decisionbox}
\textbf{Step 0: Is the section Slender (Local Buckling)?} \\
\textit{Do not assume you can use $A_n$ or $A_g$.} First check plate slenderness $b/t$ against the yield limit $\lambda_{ey}$ (Table 6.2.4).
\begin{itemize}
    \item \textbf{IF $b/t \le \lambda_{ey}$ (Non-Slender):} Use Gross Area. $A_e = A_g$.
    \item \textbf{IF $b/t > \lambda_{ey}$ (Slender):} Local buckling reduces capacity. \textbf{Calculate Effective Width $b_e$ and Effective Area $A_e$} before proceeding.
        \[ b_e = b \left( \frac{\lambda_{ey}}{b/t} \right) \quad \text{and} \quad A_e = A_g - \sum (b - b_e)t \]
\end{itemize}
\end{decisionbox}
\begin{itemize}
    \item \textbf{Purpose:} Calculates the "squash load" of the cross-section, assuming it does not buckle.
    \item \textbf{Formula:} $\phi N_s = \phi k_f A_e f_y$
    \item \textbf{Calculation:}
        \begin{align*}
            \phi N_s &= (0.9) \times (k_f = \underline{\hspace{1.5cm}}) \times (A_e = \underline{\hspace{2.5cm}}\,\text{mm}^2) \times (f_y = \underline{\hspace{2cm}}\,\text{MPa}) \\
            &= \underline{\hspace{4cm}} \text{ kN} \quad \rightarrow \quad \textbf{(Result C1)}
        \end{align*}
\end{itemize}
\begin{notebox}
This value, `Result C1`, is the maximum possible compression capacity. The member buckling check in Part B will determine how much this capacity is reduced by slenderness. The nominal section capacity, $N_s = k_f A_e f_y$, will be used in the next part.
\end{notebox}

%----------------------------------------------------------------------------------------
%   PART B: MEMBER CAPACITY (BUCKLING)
%----------------------------------------------------------------------------------------

\subsection{Part B: Member Capacity Check ($\phi N_c$)}

\begin{decisionbox}
\textbf{Step 1: Determine the Governing Buckling Axis.} \\
A column always buckles about its most slender axis. Calculate the slenderness ratio ($l_e/r$) for both axes. The \textbf{LARGER} slenderness ratio governs the design.
\begin{itemize}
    \item \textbf{Effective Length Factors ($k_e$):}
        \begin{itemize}
            \item \textbf{Braced Member (No Sway):} Use Table 4.6.3.2 ($0.5 \le k_e \le 1.0$).
            \item \textbf{Sway Member (Portal Frame/Cantilever):} Use Table 4.6.3.1 or Chart ($k_e \ge 1.0$, typically $1.2$--$2.0+$).
            \item \textbf{Sway Warning:} If part of a Portal Frame, $k_e$ is likely $> 1.0$ (Sway). Do not use $k_e=1.0$ unless fully braced.
            \item For x-axis buckling: $k_{ex} = \underline{\hspace{1.5cm}}$
            \item For y-axis buckling: $k_{ey} = \underline{\hspace{1.5cm}}$
        \end{itemize}
    \item \textbf{Effective Lengths ($l_e = k_e L$):}
        \begin{itemize}
            \item $l_{ex} = k_{ex} L = (\underline{\hspace{1.5cm}}) \times (\underline{\hspace{2cm}}) = \underline{\hspace{2.5cm}}$ mm
            \item $l_{ey} = k_{ey} L = (\underline{\hspace{1.5cm}}) \times (\underline{\hspace{2cm}}) = \underline{\hspace{2.5cm}}$ mm
        \end{itemize}
    \item \textbf{Slenderness Ratios ($l_e/r$):}
        \begin{itemize}
            \item x-axis: $l_{ex}/r_x = (\underline{\hspace{2.5cm}}) / (\underline{\hspace{2cm}}) = \underline{\hspace{2cm}}$
            \item y-axis: $l_{ey}/r_y = (\underline{\hspace{2.5cm}}) / (\underline{\hspace{2cm}}) = \underline{\hspace{2cm}}$
        \end{itemize}
    \item \textbf{Multiple Restraints:} If the column has intermediate braces (e.g., at 3m, 6m), it divides into segments. Calculate $l_e/r$ for \textbf{each segment} (using local $L$ and local $k_e$) and use the \textbf{maximum} value found.
    \item \textbf{Restraint Force:} To count as a valid restraint, a brace must be able to resist $2.5\%$ of the axial force in the column ($N^*$).
\end{itemize}
\textbf{The governing (larger) slenderness ratio is \underline{\hspace{2.5cm}}. Use the corresponding $l_e$ and $r$ values for all subsequent calculations.}
\end{decisionbox}

\subsubsection*{CHECK 2: Member Capacity (Buckling) (Clause 6.3.3)}
\begin{itemize}
    \item \textbf{Step a) Calculate Modified Member Slenderness ($\lambda_n$):}
        \[ \lambda_n = \frac{l_e}{r} \sqrt{k_f} \sqrt{\frac{f_y}{250}} = \left(\frac{\underline{\hspace{2.5cm}}}{\underline{\hspace{2.5cm}}}\right) \sqrt{\underline{\hspace{1cm}}} \sqrt{\frac{\underline{\hspace{1.5cm}}}{250}} = \underline{\hspace{2.5cm}} \]
    
    \item \textbf{Step b) Calculate Slenderness Reduction Factor ($\alpha_c$):}
        \begin{notebox}
        This is the critical step that accounts for buckling. It requires finding the member section constant ($\alpha_b$) first.
        \end{notebox}
        \begin{itemize}
            \item Look up Member Section Constant ($\alpha_b$) from \textbf{Table 6.3.3(1)} based on the member type and buckling axis. $\alpha_b = \underline{\hspace{1.5cm}}$
            \item \textbf{Residual Stress Category (Table 6.2.4):}
                \begin{itemize}
                    \item Hot Rolled (RHS, SHS, UB): SR or HR columns.
                    \item Welded (Box, I-beam): LW (Light) or HW (Heavy).
                    \item \textbf{Exam Tip:} Unless ``Stress Relieved'', assume HW for thick welded plates ($>16$mm) or complex shapes.
                    \item \textbf{Modification Warning:} If you weld plates to a Hot Rolled section (retrofit), it becomes a \textbf{Welded Section} (HW or LW). You lose the favorable ``Hot Rolled'' buckling curve.
                \end{itemize}
            \item Calculate intermediate factor $\eta$ (eta):
                \[ \eta = 0.00326 (\lambda_n - 13.5) = 0.00326 (\underline{\hspace{2cm}} - 13.5) = \underline{\hspace{2.5cm}} \quad (\text{Note: } \eta \ge 0) \]
            \item Calculate intermediate factor $\xi$ (xi):
                \[ \xi = \frac{(\lambda_n/90)^2 + 1 + \eta}{2(\lambda_n/90)^2} = \underline{\hspace{2.5cm}} \]
            \item Calculate $\alpha_c$ using the formula from Clause 6.3.3:
                \[ \alpha_c = \xi \left[ 1 - \sqrt{1 - \left(\frac{90}{\xi \lambda_n}\right)^2} \right] = \underline{\hspace{2.5cm}} \]
        \end{itemize}
    \item \textbf{Step c) Calculate Member Buckling Capacity:}
        \begin{align*}
            \phi N_c &= \phi \alpha_c N_s = (0.9) \times (\alpha_c) \times (N_s \text{ from Part A}) \\
            &= (0.9) \times (\underline{\hspace{1.5cm}}) \times (\underline{\hspace{2.5cm}}\,\text{kN}) \\
            &= \underline{\hspace{4cm}} \text{ kN} \quad \rightarrow \quad \textbf{(Result C2)}
        \end{align*}
\end{itemize}

%----------------------------------------------------------------------------------------
%   PART C: FINAL CONCLUSION
%----------------------------------------------------------------------------------------
\subsection{Part C: Final Conclusion for the Compression Member}
\begin{finalbox}{red}
The design compression capacity is the lesser of the section (crushing) capacity and the member (buckling) capacity.
\begin{itemize}
    \item Section Capacity ($\phi N_s$) = \textbf{Result C1} = \underline{\hspace{3cm}} kN
    \item Member Capacity ($\phi N_c$) = \textbf{Result C2} = \underline{\hspace{3cm}} kN
\end{itemize}
\textbf{The Design Capacity of the Compression Member is $\phi N_c = \min(\text{Result C1}, \text{Result C2}) = \underline{\hspace{4cm}}$ kN.}
\end{finalbox}

\begin{sanitybox}
\textbf{Quick checks before moving on:}
\begin{itemize}
    \item Is $\alpha_c < 1.0$? (It must be --- buckling always reduces capacity)
    \item Is $\phi N_c < \phi N_s$? (Usually yes for slender columns)
    \item Is your answer in kN (not N or MN)?
\end{itemize}
\end{sanitybox}

\begin{answerbox}
\textbf{Copy this structure into your exam answer:}
\begin{enumerate}
    \item Governing combination: \blank{4cm} $\Rightarrow$ $N^* = $ \blank{2cm} kN
    \item Effective lengths: $l_{ex} = $ \blank{2cm} mm, $l_{ey} = $ \blank{2cm} mm
    \item Governing slenderness: $(l_e/r)_{gov} = $ \blank{2cm} about \blank{2cm}-axis
    \item $\lambda_n = $ \blank{2cm}, $\alpha_b = $ \blank{1.5cm}, $\alpha_c = $ \blank{1.5cm}
    \item Section capacity: $\phi N_s = $ \blank{2cm} kN
    \item Member capacity: $\phi N_c = \phi \alpha_c N_s = $ \blank{2cm} kN
    \item Check: $N^* = $ \blank{2cm} kN $\le \phi N_c = $ \blank{2cm} kN $\Rightarrow$ \textbf{PASS / FAIL}
\end{enumerate}
\end{answerbox}

\newpage

%========================================================================================
%   END OF SECTION 4
%========================================================================================

%========================================================================================
%   SECTION 5: FLEXURAL SYSTEM (BEAM) WORKFLOW
%========================================================================================

\section{Flexural System (Beam) Workflow}

\begin{strategybox}
A beam must be checked for three primary failure modes:
\begin{enumerate}
    \item \textbf{Bending Capacity ($M^*$ vs $\phi M_b$):} Does the beam have enough strength to resist the maximum moment? This can be limited by either the cross-section's strength or by overall buckling (LTB).
    \item \textbf{Shear Capacity ($V^*$ vs $\phi V_u$):} Is the web strong enough to resist being sliced, typically near the supports?
    \item \textbf{Web Bearing/Buckling ($R^*$ vs $\phi R_b$):} Can the web handle concentrated forces from supports or point loads without crushing or buckling locally?
\end{enumerate}
This workflow will guide you through all necessary checks.
\end{strategybox}

\begin{checklistbox}
\textbf{If the question says... then you MUST include...}
\begin{itemize}
    \item \kword{unbraced} / \kword{restraint} / \kword{LTB} $\Rightarrow$ Step 2--3 (identify compression flange + LTB calc)
    \item \kword{point load} / \kword{support reaction} / \kword{bearing} $\Rightarrow$ Part C (web bearing/buckling)
    \item \kword{high shear} / \kword{near support} $\Rightarrow$ Part B (shear), then check if $V^* > 0.6\phi V_u$ for interaction
    \item \kword{continuous} / \kword{hogging} $\Rightarrow$ Critical flange is BOTTOM at supports
    \item \kword{deflection} / \kword{serviceability} $\Rightarrow$ Part E using unfactored loads
    \item \kword{cantilever} $\Rightarrow$ Load height factor $k_l$ changes; bottom flange critical for tip loads
\end{itemize}
\end{checklistbox}

\begin{dobox}
\textbf{Before you touch capacity:}
\begin{enumerate}
    \item Write the governing ULS combo and compute $M^*_{max}$ and $V^*_{max}$:
    
    $M^* = $ \blank{3cm} kNm at \blank{3cm} \quad $V^* = $ \blank{3cm} kN at \blank{3cm}
    \item Sketch BMD/SFD and circle the critical locations.
    \item At the critical section: is it \textbf{sagging} (+M, top in compression) or \textbf{hogging} (-M, bottom in compression)? \blank{3cm}
    \item Which flange is in compression? \blank{4cm}
    \item Is that compression flange braced along the segment? \underline{YES / NO}
\end{enumerate}
\end{dobox}

%----------------------------------------------------------------------------------------
%   PART A: BENDING CAPACITY
%----------------------------------------------------------------------------------------

\subsection{Part A: Bending Capacity Check ($M^* \le \phi M_b$)}
\begin{tcolorbox}[colback=gray!5,colframe=gray!50,title=Phase 1: Setup \& Data Extraction]
    \begin{itemize}
        \item \textbf{Member/Grade:} \underline{\hspace{8cm}}
        \item \textbf{Span ($L_{span}$) \& Bracing Details:} \underline{\hspace{6cm}}
        \item \textbf{Properties from Tables (AS 4100 \& Steel Handbook):}
        \begin{itemize}
            \item Capacity Factor (Bending) $\phi = 0.9$
            \item Yield Stress $f_y = \underline{\hspace{2cm}}$ MPa
            \item Plastic Section Modulus $S_x = \underline{\hspace{2cm}}$ mm$^3$
            \item Effective Section Modulus $Z_{ex} = \underline{\hspace{2cm}}$ mm$^3$
            \item Properties for LTB calc: $I_y, J, I_w$
        \end{itemize}
    \end{itemize}
\end{tcolorbox}

\begin{decisionbox}
\textbf{Step 1: Determine the Section Moment Capacity ($M_s$) by checking for COMPACTNESS.} \\
Check the plate slenderness ($\lambda_e$) for the flange and web against the plasticity limit ($\lambda_{ep}$) and yield limit ($\lambda_{ey}$) from Table 5.2.

\begin{itemize}
    \item \textbf{IF Compact ($\lambda_e \le \lambda_{ep}$):}
        \begin{itemize}
            \item The section can reach its full plastic moment. Use the \textbf{Plastic Section Modulus ($S_x$)}.
            \item Nominal Section Capacity: $M_s = f_y S_x = \underline{\hspace{4cm}}$ kNm.
        \end{itemize}
    \item \textbf{IF Non-Compact ($\lambda_{ep} < \lambda_e \le \lambda_{ey}$):}
        \begin{itemize}
            \item The section capacity is limited by local buckling. Use the \textbf{Effective Section Modulus ($Z_{ex}$)}.
            \item Nominal Section Capacity: $M_s = f_y Z_{ex} = \underline{\hspace{4cm}}$ kNm.
        \end{itemize}
    \item \textbf{IF Slender ($\lambda_e > \lambda_{ey}$):}
        \begin{itemize}
            \item \textbf{Do NOT use $S_x$ or tabulated $Z_{ex}$ for custom/slender elements.} Local buckling controls.
            \item Calculate \textbf{effective section properties} using Effective Widths (Clause 5.2.5): obtain effective areas, then $I_{eff}$.
            \item \textbf{CRITICAL STEP:} When calculating $I_{eff}$, you must find the \textbf{NEW Neutral Axis} ($\bar{y}_{eff}$) using the effective areas. The N.A. shifts away from the buckled part.
            \item Calculate $Z_{e,comp} = I_{eff} / y_{c,eff}$ and $Z_{e,tens} = I_{eff} / y_{t,eff}$.
            \item Nominal Section Capacity: $M_s = f_y \times \min(Z_{e,comp}, Z_{e,tens}) = \underline{\hspace{4cm}}$ kNm.
        \end{itemize}
\end{itemize}
\end{decisionbox}

\begin{decisionbox}
\textbf{Step 2a: Identify Critical Flange (Compression Side)}
\begin{itemize}
    \item \textbf{Sagging (+M):} Top flange is critical. Restraints must brace the top.
    \item \textbf{Hogging (-M):} Bottom flange is critical. Restraints must brace the bottom.
    \item \textbf{Cantilever (tip load):} Bottom flange is critical (usually).
    \item \textit{Warning:} If restraints are on the \textbf{tension} flange, they are ineffective ($L_e =$ full unbraced length).
\end{itemize}
\end{decisionbox}

\begin{decisionbox}
\textbf{Step 2: Check for Lateral Torsional Buckling (LTB).} \\
Is the beam segment under consideration \textbf{fully braced} against sideways movement and twisting?

\textit{Note: A restraint must prevent \textbf{both} lateral displacement \textbf{and} twist of the compression flange to be considered `full'.}

\begin{itemize}
    \item \textbf{IF YES (Fully Braced):} The member cannot buckle. The design capacity is the design section capacity.
        \begin{itemize}
            \item \textbf{Design Bending Capacity $\phi M_b = \phi M_s = \underline{\hspace{4cm}}$ kNm.}
            \item[\ding{226}] \textbf{Bending check is complete. Proceed to Part B: Shear Capacity.}
        \end{itemize}
    \item \textbf{IF NO (Unbraced or Partially Braced):} The member can buckle. The capacity must be reduced for LTB.
        \begin{itemize}
            \item[\ding{226}] \textbf{Proceed to Step 3: LTB Calculation.}
        \end{itemize}
\end{itemize}
\end{decisionbox}

\subsubsection*{Step 3: LTB Calculation for Unbraced Beams (Clause 5.6)}
\begin{itemize}
    \item \textbf{a) Determine Effective Length ($l_e$):} $l_e = k_t k_l k_r L$. Use Tables 5.6.3(1-3).
        \begin{itemize}
            \item $L$ (length between braces) = $\underline{\hspace{2cm}}$ mm. $l_e = \underline{\hspace{3cm}}$ mm.
            \item \textbf{Load Height ($k_l$):} ``Load on Top Flange'' is \textbf{destabilizing} ($k_l=1.4$) for gravity beams; \textbf{stabilizing} ($k_l<1.0$) for cantilevers (load on tension side). Check Table 5.6.3.
            \item \textbf{Continuous Beams (Hogging Support):}
                \begin{itemize}
                    \item \textbf{Option 1 (Safe):} $L=$ distance between physical bottom flange braces (e.g., column to column).
                    \item \textbf{Option 2 (Advanced):} $L=$ distance from support to point of inflection (only if the support has ``Full'' or ``Partial'' restraint; check Clause 5.6.4).
                \end{itemize}
        \end{itemize}
    \item \textbf{b) Calculate Elastic Buckling Moment ($M_{oa}$):} (Clause 5.6.1.1)
        \[ M_{oa} = \sqrt{\left(\frac{\pi^2 E I_y}{l_e^2}\right) \left( G J + \frac{\pi^2 E I_w}{l_e^2} \right)} = \underline{\hspace{3cm}} \text{ kNm} \]
    \item \textbf{c) Calculate Slenderness Reduction Factor ($\alpha_s$):}
        \[ \alpha_s = 0.6 \left[ \sqrt{\left(\frac{M_s}{M_{oa}}\right)^2 + 3} - \frac{M_s}{M_{oa}} \right] = \underline{\hspace{2.5cm}} \]
    \item \textbf{d) Determine Moment Modification Factor ($\alpha_m$):} (Table 5.6.1)
        \begin{itemize}
            \item Based on the moment diagram shape between braces. $\alpha_m = \underline{\hspace{1.5cm}}$
            \item \textbf{Sway Frames:} Assume $\alpha_m = 1.0$ (uniform moment) unless rigorous analysis proves otherwise. Do not use the shape bonus.
        \end{itemize}
    \item \textbf{e) Calculate Member Bending Capacity ($\phi M_b$):}
        \begin{align*}
            \phi M_b &= \phi \alpha_m \alpha_s M_s \\
            &= (0.9) \times (\underline{\hspace{1.5cm}}) \times (\underline{\hspace{1.5cm}}) \times (\underline{\hspace{2.5cm}}\,\text{kNm}) \\
            &= \underline{\hspace{4cm}} \text{ kNm}
        \end{align*}
\end{itemize}
\begin{tcolorbox}[colback=green!5, colframe=green!75, title=\textbf{Bending Capacity Result}]
    The governing Design Bending Capacity is $\phi M_b = \underline{\hspace{3cm}}$ kNm.
\end{tcolorbox}

\newpage
%----------------------------------------------------------------------------------------
%   PART B: SHEAR CAPACITY
%----------------------------------------------------------------------------------------

\subsection{Part B: Shear Capacity Check ($V^* \le \phi V_u$)}
\begin{tcolorbox}[colback=gray!5,colframe=gray!50,title=Data Extraction for Shear]
\begin{itemize}
    \item Web Yield Stress $f_{yw} = \underline{\hspace{2cm}}$ MPa
    \item Web depth (clear of fillets) $d_1 = \underline{\hspace{2cm}}$ mm
    \item Web thickness $t_w = \underline{\hspace{2cm}}$ mm
\end{itemize}
\end{tcolorbox}

\begin{notebox}
\textbf{High Shear Warning (Moment--Shear Interaction):} If $V^* > 0.6 \times (\text{Design Shear Capacity})$, the Moment Capacity must be reduced (apply AS 4100 bending--shear interaction provisions, commonly taught with Clause 5.12).
\begin{itemize}
    \item \textbf{What to do next:} Apply the reduced moment capacity formula from your notes/clause reference.
    \item Reduce $M_s$ if \textbf{section capacity} governs, or reduce $M_b$ if \textbf{LTB/member capacity} governs.
\end{itemize}
\end{notebox}

\begin{decisionbox}
\textbf{Does the web require a shear buckling check?} \\
A thin web may buckle before it yields. Check the slenderness ratio: Is $\frac{d_1}{t_w} \le \frac{82}{\sqrt{f_{yw} / 250}}$?
\begin{itemize}
    \item Web slenderness $\frac{d_1}{t_w} = \underline{\hspace{2cm}}$. Limit = $\underline{\hspace{2cm}}$.
\end{itemize}
\begin{itemize}
    \item \textbf{IF NO (Ratio is greater than limit):} The web is slender and will buckle.
        \begin{itemize}
            \item[\ding{226}] \textbf{Proceed to the Shear Buckling calculation below.}
        \end{itemize}
    \item \textbf{IF YES (Ratio is less than or equal to limit):} The web is stocky and will yield.
        \begin{itemize}
            \item Shear Area $A_w \approx d \times t_w = \underline{\hspace{2.5cm}}$ mm$^2$.
            \item Nominal Shear Yield Capacity $V_u = V_v = 0.6 f_{yw} A_w = \underline{\hspace{3cm}}$ kN.
            \item \textbf{Design Shear Capacity $\phi V_u = 0.9 \times V_v = \underline{\hspace{3cm}}$ kN.}
            \item[\ding{226}] \textbf{Shear check is complete. Proceed to Part C: Local Web Checks.}
        \end{itemize}
\end{itemize}
\end{decisionbox}

\subsubsection*{Shear Buckling Calculation (Clause 5.11.3)}
\begin{itemize}
    \item Shear Yield Capacity (from above) $V_v = \underline{\hspace{3cm}}$ kN.
    \item Shear Buckling Factor $\alpha_v = \left[ \frac{82}{(d_1/t_w)\sqrt{f_{yw}/250}} \right]^2 = \underline{\hspace{2.5cm}}$.
    \item Nominal Shear Buckling Capacity $V_u = V_b = \alpha_v V_v = \underline{\hspace{3cm}}$ kN.
    \item \textbf{Design Shear Capacity $\phi V_u = 0.9 \times V_b = \underline{\hspace{3cm}}$ kN.}
\end{itemize}
\begin{tcolorbox}[colback=gray!5, colframe=gray!50, sharp corners, center]
\textbf{Next: Proceed to Part C (Local Web Checks).}
\end{tcolorbox}

%----------------------------------------------------------------------------------------
%   PART C: WEB BEARING / BUCKLING
%----------------------------------------------------------------------------------------
\subsection{Part C: Local Web Checks at Supports/Point Loads ($R^* \le \phi R_b$)}
\begin{itemize}
    \item \textbf{CHECK 1: Web Bearing (Yielding) Capacity (Clause 5.13.3)}
        \begin{itemize}
            \item Effective bearing width $b_{bf} = b_s + 5t_f = \underline{\hspace{2.5cm}}$ mm
            \item $\phi R_{by} = 0.9 \times (1.25 b_{bf} t_w f_{yw}) = \underline{\hspace{3cm}}$ kN
        \end{itemize}
    \item \textbf{CHECK 2: Web Buckling Capacity (Clause 5.13.4)}
        \begin{itemize}
             \item $\phi R_{bb} = 0.9 \times (\alpha_c (1.25 b_{bf} t_w f_{yw}))$
             \item Where $\alpha_c$ is calculated for a strut of length $d_1$.
        \end{itemize}
\end{itemize}
\begin{notebox}
For most problems, Web Bearing (Yielding) is the main check required unless the web is very slender or the bearing length $b_s$ is very small.
\end{notebox}
\begin{decisionbox}
\textbf{What do you do if a local web check FAILS?}
\begin{itemize}
    \item \textbf{IF all local web checks PASS:} Proceed to Part E (Serviceability) then Part F (Final Conclusion).
    \item \textbf{IF Web Bearing/Buckling FAILS:} Proceed to Part D (Stiffeners), then re-check, then proceed to Part E and Part F.
\end{itemize}
\end{decisionbox}

%----------------------------------------------------------------------------------------
%   PART D: STIFFENERS (FIX)
%----------------------------------------------------------------------------------------
\subsection{Part D: Load Bearing Stiffeners (The Fix)}
\begin{strategybox}
If Web Bearing fails, add stiffeners. They act as a cruciform column.
\end{strategybox}
\begin{itemize}
    \item \textbf{Effective Web Length:} The stiffeners recruit a strip of web: $L_{web} \approx 25 t_w$ (or check specific clause).
    \item \textbf{Total Area ($A_s$):} Area of stiffeners + area of effective web.
    \item \textbf{Capacity:} Treat as a compression member (Section 4). $L_e \approx 0.7 d_1$ (often used).
    \item \textbf{Weld Design:} The weld must transfer load from stiffener to web. Design force $\approx N^* \times (A_{stiff} / A_{total})$.
\item \textbf{Next:} After sizing stiffeners and welds, re-check Part C local web checks, then proceed to Part E and Part F.
\end{itemize}

\subsection{Part E: Serviceability (Deflection) Checks}
\begin{strategybox}
Serviceability checks use \textbf{service/unfactored} combinations. Deflection limits depend on the problem statement; if not given, use common limits (e.g., $L/250$ to $L/360$ for floors/roofs). \textbf{If not specified, state the assumed limit clearly before checking} (markers like seeing the assumption written).
\end{strategybox}
\begin{itemize}
    \item \textbf{Step 1:} Determine service load (e.g., $G + Q$) and span $L$.
    \item \textbf{Step 2:} Compute maximum deflection $\delta_{max}$ using appropriate formula.
    \item \textbf{Step 3:} Check $\delta_{max} \le \delta_{allow}$ (e.g., $\delta_{allow} = L/250$).
    \item \textbf{Next:} Proceed to Part F (Final Conclusion).
\end{itemize}

\subsection{Part F: Final Conclusion for the Flexural System}
\begin{finalbox}{green}
To be adequate, the beam must satisfy all relevant checks.
\begin{itemize}
    \item Bending Check: $M^* = \underline{\hspace{2cm}}$ kNm $\le \phi M_b = \underline{\hspace{2cm}}$ kNm \quad (\textbf{PASS / FAIL})
    \item Shear Check: $V^* = \underline{\hspace{2cm}}$ kN $\le \phi V_u = \underline{\hspace{2cm}}$ kN \quad (\textbf{PASS / FAIL})
    \item Web Bearing Check: $R^* = \underline{\hspace{2cm}}$ kN $\le \phi R_{by} = \underline{\hspace{2cm}}$ kN \quad (\textbf{PASS / FAIL})
    \item Serviceability (Deflection) Check: $\delta_{max} = \underline{\hspace{2cm}}$ mm $\le \delta_{allow} = \underline{\hspace{2cm}}$ mm \quad (\textbf{PASS / FAIL})
\end{itemize}
\textbf{Overall Conclusion: The beam \underline{IS / IS NOT} adequate.}
\end{finalbox}

\begin{sanitybox}
\textbf{Before you finish:}
\begin{itemize}
    \item Did you check LTB if the compression flange was unbraced? (Most common omission!)
    \item Is $\phi M_b \le \phi M_s$? (LTB always reduces capacity, never increases it)
    \item For high shear ($V^* > 0.6\phi V_u$), did you reduce $M_b$?
    \item Units: Moments in kNm, forces in kN, deflections in mm.
\end{itemize}
\end{sanitybox}

\begin{answerbox}
\textbf{Copy this structure into your exam answer:}
\begin{enumerate}
    \item Governing combination: \blank{4cm}
    \item Actions: $M^*_{max} = $ \blank{2cm} kNm, $V^*_{max} = $ \blank{2cm} kN, $R^* = $ \blank{2cm} kN
    \item Section class: \blank{3cm} $\Rightarrow$ $M_s = f_y \times $ \blank{2cm} $ = $ \blank{2cm} kNm
    \item LTB check: Compression flange is \blank{3cm}, braced? \underline{YES / NO}
    \item If unbraced: $\alpha_m = $ \blank{1.5cm}, $\alpha_s = $ \blank{1.5cm} $\Rightarrow \phi M_b = $ \blank{2cm} kNm
    \item Shear: $\phi V_u = $ \blank{2cm} kN; Is $V^* > 0.6\phi V_u$? \underline{YES / NO}
    \item Web bearing: $\phi R_{by} = $ \blank{2cm} kN \quad (\textbf{PASS / FAIL})
    \item Final: Bending \textbf{PASS/FAIL}, Shear \textbf{PASS/FAIL}, Web \textbf{PASS/FAIL}
\end{enumerate}
\end{answerbox}

\newpage
%========================================================================================
%   END OF SECTION 5
%========================================================================================

%========================================================================================
%   SECTION 6: COMBINED ACTIONS (BEAM-COLUMN) WORKFLOW
%========================================================================================

\section{Combined Actions (Beam-Column) Workflow}

\begin{strategybox}
A beam-column is a member subjected to both axial compression ($N^*$) and bending ($M^*$) at the same time. These actions interact and reduce the member's capacity. We cannot simply check them in isolation ($N^* \le \phi N_c$ and $M^* \le \phi M_b$). Instead, we must use \textbf{interaction equations} to check that the combined effect is safe. There are two levels to this check:
\begin{enumerate}
    \item \textbf{Section Capacity:} Checks for yielding of the cross-section at a specific point.
    \item \textbf{Member Capacity:} Checks for overall member failure, including the interaction between column buckling and beam LTB.
\end{enumerate}
Both checks must be satisfied for the member to be adequate.
\end{strategybox}

\begin{checklistbox}
\textbf{If the question says... then you MUST include...}
\begin{itemize}
    \item \kword{eccentric} / \kword{offset load} $\Rightarrow$ Calculate $M^* = N^* \times e$ before interaction
    \item \kword{major axis} / \kword{x-axis bending} $\Rightarrow$ Out-of-plane check (Clause 8.4.4) is usually critical
    \item \kword{minor axis} / \kword{y-axis bending} $\Rightarrow$ In-plane check (Clause 8.4.2)
    \item \kword{biaxial} / \kword{both axes} $\Rightarrow$ CHECK 3 (Clause 8.4.5) is mandatory
    \item \kword{portal frame} / \kword{sway} $\Rightarrow$ $k_e > 1.0$ for compression capacity
\end{itemize}
\end{checklistbox}

\begin{dobox}
\textbf{Before you start interaction checks:}
\begin{enumerate}
    \item Do you have the PURE capacities already calculated?
    \begin{itemize}
        \item $\phi N_s = $ \blank{2cm} kN (Section 4, Part A)
        \item $\phi N_{cx} = $ \blank{2cm} kN, $\phi N_{cy} = $ \blank{2cm} kN (Section 4, Part B)
        \item $\phi M_{sx} = $ \blank{2cm} kNm (Section 5, Step 1)
        \item $\phi M_{bx} = $ \blank{2cm} kNm (Section 5, Step 3 if unbraced)
    \end{itemize}
    \item Design actions at critical location:
    
    $N^* = $ \blank{2cm} kN, $M_x^* = $ \blank{2cm} kNm, $M_y^* = $ \blank{2cm} kNm
    \item If moment not given: $M^* = N^* \times e = $ \blank{2cm} $\times$ \blank{2cm} $ = $ \blank{2cm} kNm
\end{enumerate}
\textbf{If you don't have these, STOP and go back to Sections 4 and 5!}
\end{dobox}

%----------------------------------------------------------------------------------------
%   PART A: SETUP & CAPACITY GATHERING
%----------------------------------------------------------------------------------------

\subsection{Part A: Setup \& Prerequisite Capacities}
\begin{tcolorbox}[colback=gray!5,colframe=gray!50,title=Phase 1: Data Extraction]
    \begin{itemize}
        \item \textbf{Member/Grade:} \underline{\hspace{8cm}}
        \item \textbf{Design Actions at critical location:}
            \begin{itemize}
                \item Design Axial Force $N^* = \underline{\hspace{2.5cm}}$ kN
                \item Design Bending Moment (major x-axis) $M_x^* = \underline{\hspace{2.5cm}}$ kNm
                    \begin{itemize}
                        \item \textit{If not given: $M^* = N^* \times e$ (e.g., $e = \text{Depth}/2 + \text{Dist}$).}
                        \item \textit{Minimum eccentricity (if no moment given): $M_{min}^* = N^* \times (L/500$ or similar per clause).}
                    \end{itemize}
                \item Design Bending Moment (minor y-axis) $M_y^* = \underline{\hspace{2.5cm}}$ kNm
            \end{itemize}
        \item \textbf{Required Capacities (Calculated from Section 4 \& 5 Workflows):}
            \begin{itemize}
                \item Design Section Compression Capacity $\phi N_s = \underline{\hspace{2cm}}$ kN
                \item Design Member Comp. Capacity (re: x-axis) $\phi N_{cx} = \underline{\hspace{2cm}}$ kN
                \item Design Member Comp. Capacity (re: y-axis) $\phi N_{cy} = \underline{\hspace{2cm}}$ kN
                \item Design Section Moment Capacity $\phi M_{sx} = \underline{\hspace{2cm}}$ kNm
                \item Design Member Moment Capacity $\phi M_{bx} = \underline{\hspace{2cm}}$ kNm
            \end{itemize}
    \end{itemize}
\end{tcolorbox}
\begin{notebox}
Before you can perform a combined actions check, you MUST have the individual design capacities for pure compression and pure bending. If these are not given, you must calculate them using the workflows in Section 4 and Section 5 first.
\end{notebox}

%----------------------------------------------------------------------------------------
%   PART B: COMBINED STRENGTH CHECKS
%----------------------------------------------------------------------------------------

\subsection{Part B: Combined Strength Interaction Checks}

\subsubsection*{CHECK 1: Section Capacity (Yielding) (Clause 8.3)}
\begin{itemize}
    \item \textbf{Purpose:} Checks for failure of the cross-section at a braced point of maximum moment and axial force.
    \item \textbf{Exam-safe note:} The interaction expressions below are a simplified workflow often used in teaching. \textbf{If the question explicitly asks for `in accordance with AS 4100 Clause X', use the exact clause equation rather than the simplified form below.}
    \item \textbf{Step a) Calculate Reduced Moment Capacity ($M_r$):}
        \begin{itemize}
            \item This is the section's moment capacity, reduced by the presence of the axial load.
            \item \textbf{Formula (uniaxial bending):} $M_{rx} = M_{sx} (1 - N^*/N_s)$
            \item $M_{rx} = (\underline{\hspace{2cm}}\,\text{kNm}) \times (1 - \frac{\underline{\hspace{2cm}}\,\text{kN}}{\underline{\hspace{2cm}}\,\text{kN}}) = \underline{\hspace{3cm}}$ kNm
        \end{itemize}
    \item \textbf{Step b) Perform the Interaction Check:}
        \begin{itemize}
            \item \textbf{Formula:} $\frac{M_x^*}{\phi M_{rx}} \le 1.0$
            \item $\frac{\underline{\hspace{2.5cm}}\,\text{kNm}}{0.9 \times \underline{\hspace{2.5cm}}\,\text{kNm}} = \underline{\hspace{2.5cm}}$
        \end{itemize}
\end{itemize}
\begin{tcolorbox}[colback=yellow!10!white, colframe=yellow!80!black, fonttitle=\bfseries]
    \textbf{Section Capacity Check Result:} Is the Interaction Value $\le 1.0$? \quad \textbf{\underline{YES / NO}}
\end{tcolorbox}

\hrule\vspace{1em}
\subsubsection*{CHECK 2: Member Capacity (Buckling) (Clause 8.4)}
\begin{itemize}
    \item \textbf{Purpose:} Checks for overall member failure, combining column buckling with beam buckling (LTB). For I-sections, the out-of-plane check is often critical.
    \item \textbf{Exam-safe note:} \textbf{If the question explicitly asks for `in accordance with AS 4100 Clause X', use the exact clause equation rather than the simplified form below.}
\end{itemize}
\begin{decisionbox}
\textbf{Which Member Capacity check is required?} \\
The standard specifies checks for in-plane, out-of-plane, and biaxial bending. For a typical exam problem involving an I-section bent about its major axis, the \textbf{Out-of-Plane} check is usually the governing one.

\begin{itemize}
    \item \textbf{IF the beam-column is bent about its major axis (x-axis)...}
        \begin{itemize}
            \item[\ding{226}] \textbf{Perform the Out-of-Plane Capacity Check (Clause 8.4.4.1)}
        \end{itemize}
    \item \textbf{IF the beam-column is bent about its minor axis (y-axis)...}
        \begin{itemize}
            \item[\ding{226}] \textbf{Perform the In-Plane Capacity Check (Clause 8.4.2.2)}
        \end{itemize}
    \item \textbf{IF you have significant bending about BOTH axes ($M_x^*$ and $M_y^*$)...}
        \begin{itemize}
            \item[\ding{226}] \textbf{Also perform CHECK 3: Biaxial Member Stability (Clause 8.4.5).}
        \end{itemize}
\end{itemize}
\end{decisionbox}

\paragraph{Out-of-Plane Member Capacity Check (Clause 8.4.4.1):}
\begin{itemize}
    \item \textbf{Interaction Formula:} $\left( \frac{M_x^*}{\phi M_{bx}} \right)^2 + \left( \frac{N^*}{\phi N_{cy}} \right)^2 \le 1.0$
    \item \textit{Backup for ``FIND CAPACITY'' questions (quick linear reduction):}
    \[
        \phi M_{ox} \approx \phi M_{bx}\left(1-\frac{N^*}{\phi N_{cy}}\right)
        \qquad \text{(i.e., } M_{reduced}=M_{original}(1-n)\text{)}
    \]
    \item \textbf{Calculation:}
        \begin{align*}
            \text{Moment Ratio} &= \frac{M_x^*}{\phi M_{bx}} = \frac{\underline{\hspace{2.5cm}}}{\underline{\hspace{2.5cm}}} = \underline{\hspace{2.5cm}} \\
            \text{Compression Ratio} &= \frac{N^*}{\phi N_{cy}} = \frac{\underline{\hspace{2.5cm}}}{\underline{\hspace{2.5cm}}} = \underline{\hspace{2.5cm}} \\
            \text{Interaction Value} &= (\text{Moment Ratio})^2 + (\text{Compression Ratio})^2 \\
            &= (\underline{\hspace{2cm}})^2 + (\underline{\hspace{2cm}})^2 = \underline{\hspace{2.5cm}}
        \end{align*}
\end{itemize}
\begin{tcolorbox}[colback=yellow!10!white, colframe=yellow!80!black, fonttitle=\bfseries]
    \textbf{Out-of-Plane Member Check Result:} Is the Interaction Value $\le 1.0$? \quad \textbf{\underline{YES / NO}}
\end{tcolorbox}

\hrule\vspace{1em}
\subsubsection*{CHECK 3: Biaxial Member Stability (Clause 8.4.5)}
\begin{itemize}
    \item \textbf{Purpose:} Required if you have $N^*, M_x^*, \text{and } M_y^*$.
    \item \textbf{Simplified Interaction Formula:}
        \[
            \left(\frac{M_x^*}{\phi M_{cx}}\right)^{1.4} + \left(\frac{M_y^*}{\phi M_{iy}}\right)^{1.4} \le 1.0
        \]
        \textit{Note: This assumes axial load ($N^*$) is already accounted for inside the reduced capacities $M_{cx}$ and $M_{iy}$.}
    \item \textbf{Alternative (Linear Conservative):} Sum the usage ratios from Checks 1 and 2.
\end{itemize}

%----------------------------------------------------------------------------------------
%   PART C: FINAL CONCLUSION
%----------------------------------------------------------------------------------------
\subsection{Part C: Final Conclusion for the Beam-Column}
\begin{finalbox}{purple}
To be adequate, a beam-column must satisfy \textbf{ALL} applicable combined action checks.
\begin{itemize}
    \item Section Capacity Check: \textbf{\underline{PASS / FAIL}}
    \item Member Capacity Check: \textbf{\underline{PASS / FAIL}}
\end{itemize}
\textbf{Overall Conclusion: The member \underline{IS / IS NOT} adequate for the combined actions.}
\end{finalbox}

\begin{sanitybox}
\textbf{Interaction check sanity:}
\begin{itemize}
    \item Each ratio ($M^*/\phi M$, $N^*/\phi N$) must be $< 1.0$ individually before you even check interaction!
    \item The interaction value should be $< 1.0$ (not $< 100$ or $< 0.01$ --- check your units)
    \item Did you use $\phi N_{cy}$ (weak axis) for out-of-plane check? (Common error: using $\phi N_{cx}$)
\end{itemize}
\end{sanitybox}

\begin{answerbox}
\textbf{Copy this structure into your exam answer:}
\begin{enumerate}
    \item Actions: $N^* = $ \blank{2cm} kN, $M_x^* = $ \blank{2cm} kNm
    \item Pure capacities (from Sections 4 \& 5):
    \begin{itemize}
        \item $\phi N_s = $ \blank{2cm} kN, $\phi N_{cy} = $ \blank{2cm} kN
        \item $\phi M_{sx} = $ \blank{2cm} kNm, $\phi M_{bx} = $ \blank{2cm} kNm
    \end{itemize}
    \item Section check (Clause 8.3): $M_{rx} = M_{sx}(1 - N^*/N_s) = $ \blank{2cm} kNm
    
    $M_x^*/\phi M_{rx} = $ \blank{2cm} $\le 1.0$? \textbf{PASS / FAIL}
    \item Member check (Clause 8.4): $(M_x^*/\phi M_{bx})^2 + (N^*/\phi N_{cy})^2 = $ \blank{2cm} $\le 1.0$? \textbf{PASS / FAIL}
    \item \textbf{Conclusion: Member IS / IS NOT adequate}
\end{enumerate}
\end{answerbox}

\newpage

%========================================================================================
%   END OF SECTION 6
%========================================================================================

%========================================================================================
%   SECTION 7: ADVANCED CONNECTION --- ECCENTRIC WELDS
%========================================================================================

\section{Advanced Connection: Eccentric Welds Workflow}

\begin{strategybox}
An eccentric load on a weld group creates two effects: a direct shear force that is distributed evenly, and a torque (moment) that creates shear forces that are highest at the points farthest from the group's center. We cannot simply add these forces. We must treat them as vectors and find the resultant force at the most critical point. The weld is adequate if this maximum resultant force (per unit length) is less than the weld's design capacity. \textbf{For web welds in beams:} the weld can resist vertical shear ($V^*$) \textbf{and} a portion of the moment ($M^*$); check resultant stress vectors.
\end{strategybox}

\begin{dobox}
\textbf{Before you start weld calculations:}
\begin{enumerate}
    \item Convert all forces to \textbf{N} (not kN!) for consistency with N/mm weld capacity.
    \item Sketch the weld group and mark:
    \begin{itemize}
        \item Centroid location ($\bar{x}, \bar{y}$)
        \item Load application point
        \item Eccentricity $e = $ \blank{3cm} mm
        \item Critical corner (farthest from centroid)
    \end{itemize}
    \item Direct shear $V^* = $ \blank{3cm} N
    \item Torque $T^* = V^* \times e = $ \blank{3cm} Nmm
\end{enumerate}
\end{dobox}

%----------------------------------------------------------------------------------------
%   PART A: WELD GROUP PROPERTIES & CAPACITY
%----------------------------------------------------------------------------------------

\subsection{Part A: Weld Group Properties \& Capacity}
\begin{tcolorbox}[colback=gray!5,colframe=gray!50,title=Phase 1: Setup \& Data Extraction]
    \begin{itemize}
        \item \textbf{Weld Details:} \underline{\hspace{2cm}} mm \underline{\hspace{2cm}} (SP/GP) Fillet Welds
        \item \textbf{Load Details:} Design Shear Force $V^* = \underline{\hspace{2cm}}$ kN, Eccentricity $e = \underline{\hspace{2cm}}$ mm
        \item \textbf{Weld Metal Strength} $f_{uw} = \underline{\hspace{2cm}}$ MPa (e.g., 490 MPa for E49XX)
        \item \textbf{Capacity Factor $\phi$}: \underline{\hspace{1cm}} (e.g., 0.8 for SP, 0.6 for GP)
    \end{itemize}
\end{tcolorbox}

\subsubsection*{Step 1: Calculate the Design Weld Capacity per mm ($\phi v_w$)}
\begin{itemize}
    \item \textbf{Formula:} $\phi v_w = \phi \times (0.6 f_{uw} t_t)$
    \item Throat Thickness $t_t = 0.707 \times (\text{leg size}) = 0.707 \times (\underline{\hspace{1.5cm}}) = \underline{\hspace{2cm}}$ mm
    \item $\phi v_w = (\underline{\hspace{0.5cm}}) \times (0.6) \times (\underline{\hspace{1.5cm}}\,\text{MPa}) \times (\underline{\hspace{1.5cm}}\,\text{mm}) = \underline{\hspace{3cm}}$ N/mm
\end{itemize}

\subsubsection*{Step 2: Calculate Weld Group Geometric Properties}
\begin{itemize}
    \item \textbf{a) Find the Centroid of the Weld Group ($\bar{x}, \bar{y}$):}
    \begin{itemize}
        \item Treat the welds as lines. Find the centroid using the formula $\bar{x} = \frac{\sum L_i x_i}{\sum L_i}$.
        \item Total Length of Weld $L_{total} = \underline{\hspace{2.5cm}}$ mm
        \item Centroid coordinates: $\bar{x} = \underline{\hspace{2cm}}$ mm, $\bar{y} = \underline{\hspace{2cm}}$ mm
    \end{itemize}
    \item \textbf{b) Calculate the Polar Moment of Inertia ($I_p$):}
    \begin{itemize}
        \item \textbf{Use ONE consistent weld-group method in the exam.} A common line-method approximation is:
            \[
            I_p \approx \sum (L_i r_i^2)
            \]
            where $r_i$ is the distance from the group centroid to the centroid of weld line $i$. Units: $I_p$ in mm$^3$.
        \item \textit{Do not mix line-method and area-method inertia definitions in the same calculation.}
        \item $I_p = \underline{\hspace{4cm}}$ mm$^3$
    \end{itemize}
\end{itemize}
\begin{notebox}
\textbf{Unit consistency (CRITICAL):} Use N and mm consistently throughout weld group calculations (not kN). Convert forces to N before calculating torque and shear stresses.

\textbf{Unit sanity check (torsional shear):} $T^*$ is in Nmm and $I_p$ is in mm$^3$. So $T^*/I_p$ gives N/mm$^2$, then $(T^*/I_p)\times r$ gives N/mm (matches $v_t^*$).
\end{notebox}

%----------------------------------------------------------------------------------------
%   PART B: FORCE ANALYSIS AT CRITICAL POINT
%----------------------------------------------------------------------------------------

\subsection{Part B: Force Analysis at the Critical Point}
\begin{notebox}
The critical point is almost always one of the corners of the weld group, farthest from the centroid, where the torsional shear will be at its maximum.
\end{notebox}

\subsubsection*{Step 3: Calculate the Design Actions on the Weld Group}
\begin{itemize}
    \item \textbf{Direct Shear Force (Vertical):} $V^* = \underline{\hspace{3cm}}$ N
    \item \textbf{Design Torque:} $T^* = V^* \times e = (\underline{\hspace{2cm}}\,\text{N}) \times (\underline{\hspace{2cm}}\,\text{mm}) = \underline{\hspace{4cm}}$ Nmm
\end{itemize}

\subsubsection*{Step 4: Calculate Force Components per mm at the Critical Point}
\begin{itemize}
    \item \textbf{a) Direct Shear Component ($v_v^*$):}
        \begin{itemize}
            \item This force acts in the same direction as the applied load ($V^*$).
            \item $v_{v,y}^* = \frac{V^*}{L_{total}} = \frac{\underline{\hspace{3cm}}\,\text{N}}{\underline{\hspace{3cm}}\,\text{mm}} = \underline{\hspace{3cm}}$ N/mm (acting downwards)
            \item $v_{v,x}^* = 0$ N/mm
        \end{itemize}
    \item \textbf{b) Torsional Shear Components ($v_t^*$):}
        \begin{itemize}
            \item This force acts perpendicular to a line drawn from the centroid to the critical point.
            \item Coordinates of critical point relative to centroid: $r_x = \underline{\hspace{1.5cm}}$ mm, $r_y = \underline{\hspace{1.5cm}}$ mm
            \item Horizontal component of torsional shear: $v_{t,x}^* = \frac{T^* \times r_y}{I_p} = \frac{(\underline{\hspace{3cm}})(\underline{\hspace{1.5cm}})}{\underline{\hspace{4cm}}} = \underline{\hspace{3cm}}$ N/mm
            \item Vertical component of torsional shear: $v_{t,y}^* = \frac{T^* \times r_x}{I_p} = \frac{(\underline{\hspace{3cm}})(\underline{\hspace{1.5cm}})}{\underline{\hspace{4cm}}} = \underline{\hspace{3cm}}$ N/mm
        \end{itemize}
\end{itemize}

\subsubsection*{Step 5: Calculate the Resultant Force per mm ($v_{res}^*$)}
\begin{itemize}
    \item Combine the components vectorially. Pay close attention to directions (e.g., direct shear is down, torsional shear might be up or down).
    \item Total Vertical Force: $v_{total, y}^* = v_{v,y}^* + v_{t,y}^* = \underline{\hspace{2cm}} + \underline{\hspace{2cm}} = \underline{\hspace{2.5cm}}$ N/mm
    \item Total Horizontal Force: $v_{total, x}^* = v_{v,x}^* + v_{t,x}^* = 0 + \underline{\hspace{2cm}} = \underline{\hspace{2.5cm}}$ N/mm
    \item \textbf{Resultant Force:} $v_{res}^* = \sqrt{(v_{total, x}^*)^2 + (v_{total, y}^*)^2} = \underline{\hspace{3cm}}$ N/mm
\end{itemize}

%----------------------------------------------------------------------------------------
%   PART C: FINAL CONCLUSION
%----------------------------------------------------------------------------------------

\subsection{Part C: Final Conclusion for the Eccentric Weld}
\begin{finalbox}{teal}
The weld group is adequate if the maximum resultant force per unit length is less than the design capacity of the weld per unit length.
\begin{itemize}
    \item Maximum Resultant Force ($v_{res}^*$) = \underline{\hspace{3cm}} N/mm
    \item Design Weld Capacity ($\phi v_w$) = \underline{\hspace{3cm}} N/mm
\end{itemize}
\textbf{Check: Is $v_{res}^* \le \phi v_w$? \quad \underline{YES / NO}} \\
\textbf{Conclusion: The eccentric weld group \underline{IS / IS NOT} adequate.}
\end{finalbox}

\begin{sanitybox}
\textbf{Weld calculation sanity:}
\begin{itemize}
    \item Are ALL forces in N (not kN)? $1$ kN $= 1000$ N
    \item Is $v_{res}^*$ in N/mm? (Should be hundreds, not millions)
    \item Did you use the correct $\phi$? (0.8 for SP, 0.6 for GP)
    \item Does the vector diagram make sense? (Torsional shear perpendicular to radius)
\end{itemize}
\end{sanitybox}

\begin{answerbox}
\textbf{Copy this structure into your exam answer:}
\begin{enumerate}
    \item Weld capacity: $\phi v_w = \phi \times 0.6 f_{uw} \times t_t = $ \blank{3cm} N/mm
    \item Weld group: $L_{total} = $ \blank{2cm} mm, Centroid at $(\bar{x}, \bar{y}) = $ \blank{3cm}
    \item $I_p = $ \blank{3cm} mm$^3$
    \item Direct shear: $v_v^* = V^*/L_{total} = $ \blank{3cm} N/mm
    \item Torsional shear at critical point: $v_{t,x}^* = $ \blank{2cm}, $v_{t,y}^* = $ \blank{2cm} N/mm
    \item Resultant: $v_{res}^* = \sqrt{v_x^2 + v_y^2} = $ \blank{3cm} N/mm
    \item Check: $v_{res}^* = $ \blank{2cm} $\le \phi v_w = $ \blank{2cm} N/mm $\Rightarrow$ \textbf{PASS / FAIL}
\end{enumerate}
\end{answerbox}

\newpage

%========================================================================================
%   END OF SECTION 7
%========================================================================================

%========================================================================================
%   SECTION 8: ADVANCED CONNECTION --- BOLTS IN COMBINED ACTION
%========================================================================================

\section{Advanced Connection: Bolts in Combined Shear \& Tension}

\begin{strategybox}
Many connections, such as fin plates or brackets, subject bolts to a combination of shear ($V_f^*$) and tension ($N_{tf}^*$) simultaneously. Neither the pure shear capacity nor the pure tension capacity is sufficient on its own. We must use an \textbf{interaction formula} to check that the combined effect of these two actions is acceptable for the most critically loaded bolt.
\end{strategybox}

\begin{checklistbox}
\textbf{If the question says... then you MUST include...}
\begin{itemize}
    \item \kword{threads in shear plane} / \kword{threads included} $\Rightarrow$ Use $A_c$ (core area), not $A_o$ (shank)
    \item \kword{threads excluded} / \kword{shank in shear} $\Rightarrow$ Use $A_o$ (shank area)
    \item \kword{prying} / \kword{flexible plate} / \kword{end plate} $\Rightarrow$ Multiply tension by prying factor (1.2--1.4)
    \item \kword{long connection} / $L_j > 300$ mm $\Rightarrow$ Calculate $k_r < 1.0$
    \item \kword{bearing} / \kword{edge distance} $\Rightarrow$ Also check ply bearing capacity
\end{itemize}
\end{checklistbox}

\begin{dobox}
\textbf{Before you start bolt calculations:}
\begin{enumerate}
    \item Number of bolts in group: \blank{2cm}
    \item Which bolt is most critical? (Usually the corner bolt for eccentric loads)
    \item Actions on \textbf{most critical bolt}:
    
    $V_f^* = $ \blank{3cm} kN (shear per bolt), $N_{tf}^* = $ \blank{3cm} kN (tension per bolt)
    \item Threads in shear plane? \underline{YES / NO} $\Rightarrow$ Use $A_{eff} = $ \blank{2cm} mm$^2$
    \item Is prying relevant? \underline{YES / NO} $\Rightarrow$ If yes, multiply $N_{tf}^*$ by \blank{1.5cm}
\end{enumerate}
\end{dobox}

%----------------------------------------------------------------------------------------
%   PART A: SETUP & INDIVIDUAL CAPACITIES
%----------------------------------------------------------------------------------------

\subsection{Part A: Setup \& Individual Bolt Capacities}
\begin{tcolorbox}[colback=yellow!10!white, colframe=yellow!80!black, fonttitle=\bfseries, title=Pre-Check: Geometry (Detailing Marks)]
\begin{itemize}
    \item Minimum Pitch: $s_{min} = 2.5 d_f$
    \item Minimum Edge Distance: $e_{min} = 1.25 d_f$ (Machine cut) or $1.50 d_f$ (Hand cut) ($1.75 d_f$ for sheared edges).
\end{itemize}
\end{tcolorbox}
\vspace{0.5em}
\begin{tcolorbox}[colback=gray!5,colframe=gray!50,title=Phase 1: Data Extraction]
    \begin{itemize}
        \item \textbf{Bolt Details:} \underline{\hspace{2cm}} Grade \underline{\hspace{1.5cm}}, Diameter $d_f = \underline{\hspace{1cm}}$ mm
        \item \textbf{Connection Details:} Threads are \underline{Included / Excluded} from the shear plane.
        \item \textbf{Design Actions on the MOST CRITICAL BOLT:}
            \begin{itemize}
                \item Design Shear Force per bolt $V_f^* = \underline{\hspace{2.5cm}}$ kN
                \item Design Tension Force per bolt $N_{tf}^* = \underline{\hspace{2.5cm}}$ kN
            \end{itemize}
        \item \textbf{Properties from Tables (AS 4100 \& Steel Handbook):}
            \begin{itemize}
                \item Capacity Factor (Bolts) $\phi = 0.8$ (Table 3.4)
                \item Bolt ultimate tensile strength $f_{uf} = \underline{\hspace{2cm}}$ MPa (Table 9.3.1)
                \item Bolt tensile stress area $A_s = \underline{\hspace{2cm}}$ mm$^2$
                \item Bolt shear area (core $A_c$ or shank $A_o$) = $\underline{\hspace{2cm}}$ mm$^2$
            \end{itemize}
    \end{itemize}
\end{tcolorbox}
\begin{notebox}
For a group of bolts, you must first determine the actions on the most critically loaded bolt before using this workflow. For a simple concentric load, this is just the total load divided by the number of bolts. For an eccentric load, you may need to perform a separate analysis.
\end{notebox}

\subsubsection*{Step 1: Calculate the Design SHEAR Capacity per bolt ($\phi V_f$)}
\begin{itemize}
    \item \textbf{Formula (Clause 9.3.2.1):} $\phi V_f = \phi (0.62 f_{uf} A_{eff} k_r)$
    \item \textbf{Determine reduction factor $k_r$:}
        \begin{itemize}
            \item If connection length $L_j < 300$ mm, $k_r = 1.0$.
            \item $L_j$ is the \textbf{distance between bolt centers} from the first to the last bolt in the joint (along the line of load transfer).
            \item \textit{Example:} 1 row of 2 bolts at 60 mm spacing $\Rightarrow L_j = 60$ mm.
            \item \textit{Example:} 1 row of 10 bolts at 60 mm spacing $\Rightarrow L_j = 9 \times 60 = 540$ mm.
            \item If $L_j > 300$ mm, calculate $k_r = 1.075 - L_j/4000$.
        \end{itemize}
    \item $\phi V_f = (0.8) \times (0.62) \times (\underline{\hspace{2cm}}\,\text{MPa}) \times (\underline{\hspace{2cm}}\,\text{mm}^2) \times (k_r)$
    \item $\phi V_f = \underline{\hspace{4cm}}$ kN
\end{itemize}

\subsubsection*{Step 2: Calculate the Design TENSION Capacity per bolt ($\phi N_{tf}$)}
\begin{itemize}
    \item \textbf{Formula (Clause 9.3.2.2):} $\phi N_{tf} = \phi A_s f_{uf}$
    \item \textbf{Load Note:} If the bolts connect a flexible plate (end plate/hanger), multiply the static tension by a \textbf{Prying Factor} (typically $1.2$--$1.4$) unless the plate is very thick.
    \item $\phi N_{tf} = (0.8) \times (\underline{\hspace{2cm}}\,\text{mm}^2) \times (\underline{\hspace{2cm}}\,\text{MPa})$
    \item $\phi N_{tf} = \underline{\hspace{4cm}}$ kN
\end{itemize}

%----------------------------------------------------------------------------------------
%   PART B: COMBINED ACTION CHECK
%----------------------------------------------------------------------------------------

\subsection{Part B: Combined Action Interaction Check}

\subsubsection*{Step 3: Apply the Interaction Formula (Clause 9.3.2.3)}
\begin{itemize}
    \item \textbf{Purpose:} To ensure that the combination of applied shear and tension does not exceed the bolt's capacity.
    \item \textbf{Formula (Elliptical Interaction):}
        \[ \left( \frac{V_f^*}{\phi V_f} \right)^2 + \left( \frac{N_{tf}^*}{\phi N_{tf}} \right)^2 \le 1.0 \]
    \item \textbf{Calculation:}
        \begin{align*}
            \text{Shear Ratio} &= \frac{V_f^*}{\phi V_f} = \frac{\underline{\hspace{2.5cm}}}{\underline{\hspace{2.5cm}}} = \underline{\hspace{2.5cm}} \\
            \text{Tension Ratio} &= \frac{N_{tf}^*}{\phi N_{tf}} = \frac{\underline{\hspace{2.5cm}}}{\underline{\hspace{2.5cm}}} = \underline{\hspace{2.5cm}} \\
            \text{Interaction Value} &= (\text{Shear Ratio})^2 + (\text{Tension Ratio})^2 \\
            &= (\underline{\hspace{2cm}})^2 + (\underline{\hspace{2cm}})^2 = \underline{\hspace{2.5cm}}
        \end{align*}
\end{itemize}

%----------------------------------------------------------------------------------------
%   PART C: FINAL CONCLUSION
%----------------------------------------------------------------------------------------

\subsection{Part C: Final Conclusion for the Bolt in Combined Action}
\begin{finalbox}{violet}
The bolt is adequate if the interaction value from the combined check is less than or equal to 1.0.
\begin{itemize}
    \item Interaction Value = \underline{\hspace{3cm}}
\end{itemize}
\textbf{Check: Is the Interaction Value $\le 1.0$? \quad \underline{YES / NO}} \\
\textbf{Conclusion: The bolt \underline{IS / IS NOT} adequate for the combined actions.}
\end{finalbox}

\begin{sanitybox}
\textbf{Bolt check sanity:}
\begin{itemize}
    \item Each ratio ($V_f^*/\phi V_f$, $N_{tf}^*/\phi N_{tf}$) must be $< 1.0$ individually!
    \item Interaction value should be a number between 0 and 1 (not 0.001 or 50)
    \item Did you use the correct area? ($A_c$ if threads in shear, $A_o$ if excluded)
    \item Did you account for prying if the plate is flexible?
\end{itemize}
\end{sanitybox}

\begin{answerbox}
\textbf{Copy this structure into your exam answer:}
\begin{enumerate}
    \item Bolt: \blank{2cm} Grade \blank{1.5cm}, $d_f = $ \blank{1.5cm} mm
    \item Threads in shear plane? \underline{YES / NO} $\Rightarrow A_{eff} = $ \blank{2cm} mm$^2$
    \item Actions per critical bolt: $V_f^* = $ \blank{2cm} kN, $N_{tf}^* = $ \blank{2cm} kN
    \item Shear capacity: $\phi V_f = 0.8 \times 0.62 \times f_{uf} \times A_{eff} \times k_r = $ \blank{2cm} kN
    \item Tension capacity: $\phi N_{tf} = 0.8 \times A_s \times f_{uf} = $ \blank{2cm} kN
    \item Interaction: $(V_f^*/\phi V_f)^2 + (N_{tf}^*/\phi N_{tf})^2 = $ \blank{2cm} $\le 1.0$? \textbf{PASS / FAIL}
\end{enumerate}
\end{answerbox}

\newpage

%========================================================================================
%   END OF SECTION 8
%========================================================================================

\newpage
\appendix
\section{Common Beam Formulas (Quick Reference)}
\begin{notebox}
\textbf{Use the correct load case and boundary conditions.} These are common elastic formulas for prismatic members with constant $E I$.
\end{notebox}
\begin{itemize}
    \item \textbf{Axial Member (Tension/Compression):}
        \begin{itemize}
            \item Deformation $\delta = \frac{P L}{A E}$
        \end{itemize}
    \item \textbf{Simply Supported Beam, UDL $w$ over span $L$:}
        \begin{itemize}
            \item $M_{max} = wL^2/8$, \quad $V_{max} = wL/2$
            \item $\delta_{max} = 5wL^4/(384EI)$
        \end{itemize}
    \item \textbf{Simply Supported Beam, Point Load $P$ at midspan:}
        \begin{itemize}
            \item $M_{max} = PL/4$, \quad $V_{max} = P/2$
            \item $\delta_{max} = PL^3/(48EI)$
        \end{itemize}
    \item \textbf{Cantilever, Point Load $P$ at free end:}
        \begin{itemize}
            \item $M_{max} = PL$ (at fixed end), \quad $V_{max} = P$
            \item $\delta_{max} = PL^3/(3EI)$
        \end{itemize}
    \item \textbf{Cantilever, UDL $w$ over span $L$:}
        \begin{itemize}
            \item $M_{max} = wL^2/2$, \quad $V_{max} = wL$
            \item $\delta_{max} = wL^4/(8EI)$
        \end{itemize}
\end{itemize}

\end{document}
